\documentclass{article} %[default 8pt]
\usepackage[utf8]{inputenc}
\usepackage[top = 2.5cm, bottom = 2.5cm, left = 2.5cm, right = 2.5cm]{geometry} %set margin 

% The default setting of LaTeX is to indent new paragraphs. This is useful for articles. But not really nice for homework problem sets. The following command sets the indent to 0.
\usepackage{setspace}

%\setlength{\parindent}{0in}%

% Package to place figures where you want them.
\usepackage{float}

% The fancyhdr package let's us create nice headers.
\usepackage{fancyhdr}

%%%%%%%%%%%%%%%%%%%%%%%%%%%%%%%%%%%%%%%%%%%%%%%%
\pagestyle{fancy} % With this command we can customize the header style.

\fancyhf{} % This makes sure we do not have other information in our header or footer.

\lhead{\footnotesize Microeconomics I: Midterm Cheat Sheet}% \lhead puts text in the top left corner. \footnotesize sets our font to a smaller size.

%\rhead works just like \lhead (you can also use \chead)
\rhead{\footnotesize Hsien-Chen Chu} %<---- Fill in your lastnames.

% Similar commands work for the footer (\lfoot, \cfoot and \rfoot).
% We want to put our page number in the center.
\cfoot{\footnotesize \thepage} 

%%%%%%%%%%%%%%%%%%%%%%%%%%%%%%%%%%%%%%%%%%%%%%%%
%%%%%%%%%%%%%%%%%%%%%%%%%%%%%%%%%%%%%%%%%%%%%%%%

\begin{document}
\thispagestyle{empty}
\begin{tabular}{p{15.5cm}} % This is a simple tabular environment to align your text nicely 
{\large \bf Microeconomics I} \\
National Taiwan University \\ Fall 2020 \\ Hsien-Chen Chu \\ %information about the course and file creator
\hline % \hline produces horizontal lines.
\\
\end{tabular} % Our tabular environment ends here.

\vspace*{0.3cm}
\begin{center}
    {\Large \bf Midterm Cheat Sheet} 
    % <---- Don't forget to put in the right number
    \\
    Last Edited: [2020.11.09]
	\vspace{2mm}
	
     % NAMES GO HERE
	{\bf Hsien-Chen Chu (T09303304)} % <---- Fill in your names here!
\end{center}
\vspace{0.4cm}

%%%%%%%%%%%%%%%%%%%%%%%%%%%%%%%%%%%%%%%%%%%%%%%%
% Up until this point you only have to make minor changes for every week (Number of the homework). Your write up essentially starts here.
%%%%%%%%%%%%%%%%%%%%%%%%%%%%%%%%%%%%%%%%%%%%%%%%

\section{Preference} % #Problem Set 1
\paragraph{a} 
    Check [SM] is satisfied: $\forall$$x_1,x_2>0$, $MU_1>0$, $MU_2>0$.
    
    Check [SC] is satisfied: $\frac{\partial|MRS|}{\partial x_1}<0$ (diminishing in $x_1$) or $\frac{\partial|MRS|}{\partial x_2}>0$ (increasing in $x_2$).
    
    
\paragraph{b} 
    After checking [SM]\&[SC], while both holding,  we can link these features to our optimal choice problem.
    \\
   
\section{Choices: Find $x_1^m, x_2^m$} % #Problem Set 2
\paragraph{a}
    The maximization problem is: $\mathop{max}\limits_{x_1,x_2} u(x_1,x_2),$  $s.t.$  $p_1x_1+p_2x_2=I$. 
    
    Suppose the utility function $u(x_1,x_2)$ holds [SM]\&[SC]:
    
    The Lagrange is: $\mathop{max}\limits_{x_1,x_2} \mathcal{L}= u(x_1,x_2)+\lambda(I - p_1x_1 - p_2x_2)$. By F.O.C, we obtain $\frac{\partial \mathcal{L}/\partial x_1}{\partial \mathcal{L}/\partial x_2}=|MRS|=\frac{p_1}{p_2}$. 
    
    And then plug this relation back in given Budget Constraint: $x_1^*=x_1(p_1,p_2,I)=x_1^m; x_2^*=x_2(p_1,p_2,I)= $
    
   $ x_2^m$.
    

\paragraph{b}
    Whether satisfies "Law of Demand": check $\varepsilon_1<0 \Leftrightarrow \frac{\partial x_1}{\partial p_1}<0$.
    \\

\section{Elasticity} % #Problem Set 3
\paragraph{a}
    $\varepsilon_1 = \frac{\partial x_1}{\partial p_1}\frac{p_1}{x_1}$, if $\varepsilon_1<0$: Ordinary good $\Leftrightarrow$ satisfies L.O.D. Otherwise, Giffen good.
    


\paragraph{b} 
    $\varepsilon_{ij} = \frac{\partial x_i}{\partial p_j}\frac{p_j}{x_i}$. Suppose $x_i, x_j$ are ordinary goods: If $\varepsilon_{ij}>0$ $\Leftrightarrow$ Substitutes. Otherwise, Complements.
    
\paragraph{c}
    $\varepsilon_{iI} = \frac{\partial x_i}{\partial I}\frac{I}{x_i}$, if $\varepsilon_{iI}>0$ $\Leftrightarrow$ Normal good. Otherwise, Inferior good.
    \\
\section{Derive SE \& IE: Find $x_1^h, x_2^h$}

\paragraph{a}
    Original Choice: $e^*=(x_1^*, x_2^*)$ $\rightarrow$ $p_i$ changes: New Choice: $e'=(x_1', x_2')$
    
    Derivation (Slutsky equation by Hicksian's Methods): $e''=(x_1'', x_2'') = (x_1^h, x_2^h)$
    
\paragraph{b}
    The minimization problem is: $\mathop{min}\limits_{x_1,x_2} p_1x_1+p_2x_2,$  $s.t.$  $ \bar{u}(x_1^*,x_2^*)$. Obtain: 

    $x_1''=x_1(p_1,p_2,\bar{u})=x_1^h$
    
    $x_2''=x_2(p_1,p_2,\bar{u})=x_2^h$
    
\paragraph{c}
    Substitution Effect(SE): $x_1''-x_1^* = x_1^h -x_1^*$
    
    Income Effect(IE): $x_1'-x_1'' = x_1' -x_1^h$
    
    Total Effect = SE + IE
    \\    
\section{Intertemporal Consumption}

\paragraph{a}
    The maximization problem is: $\mathop{max}\limits_{c_1,c_2} u(c_1,c_2),$  $s.t.$  $c_1+\frac{c_2}{1+r}=I_1+\frac{I_2}{1+r}$.
\paragraph{b}
    Directly use Lagrange: obtain $(c_1^*, c_2^*)$
    
    Check: \#saver: $c_1^* < I_1$ or \#borrower: $c_1^* > I_1$
    
    Basic Assumption: Both $c_1, c_2$ are normal goods. 
    
    Discussion: SE \& IE in terms of $c_1$
\paragraph{c}
    {\bf Comparative Static Analysis: r changes}
\\

    {\bf Suppose a borrower in Period 1} 
    
    $r \uparrow _{small}$(remain a borrower): $SE<0$ (Oppor.Cost of $c_1$ increases $\Rightarrow$ less $c_1$). $IE<0$ (for a borrower: r $\uparrow$ means real I $\downarrow$) {\bf Results:} [1]$(c_1\downarrow, c_2?)$, [2]remain a borrower, [3]utility $\downarrow$
\\
    
    $r \uparrow _{large}$(become a saver): \underline{STRONG} $SE<0$ (Oppor.Cost of $c_1$ increases $\Rightarrow$ less $c_1$). $IE>0$ (already become a saver: r $\uparrow$ means real I $\uparrow$) {\bf Results:} [1]$(c_1\downarrow, c_2\uparrow)$, [2]become a saver, [3]utility $\uparrow$
    \\   
    
    
    {\bf Suppose a saver in Period 1} 
    
    $r \uparrow$: $SE<0$ (Oppor.Cost of $c_1$ increases $\Rightarrow$ less $c_1$) $IE>0$ (for a saver: r $\uparrow$ means real I $\uparrow$) {\bf Results:} [1]$(c_1?, but <I_1, c_2\uparrow)$, [2]remain a saver, [3]utility $\uparrow$

\paragraph{d}  
    {\bf Comparative Static Analysis: I changes}
\\    
    
    {\bf Suppose a saver in Period 1}
    
    $I_1 \uparrow$: $IE>0$ (both normal goods). Results: [1]$(c_1\uparrow, c_2\uparrow)$, [2]remain a saver, [3]utility $\uparrow$
    
    $I_2 \uparrow$: $IE>0$ (both normal goods). Results: [1]$(c_1\uparrow, c_2\uparrow)$, [2]remain a saver or become a borrower, [3]utility $\uparrow$ 
    \\
\section{Affect}
\paragraph{a}
    Check "variable $\alpha$'s influence" on the target function $v(x_1, x_2)$  
\paragraph{b}    
    Method: take partial $\frac{\partial v(x_1, x_2)}{\partial \alpha}$ and verify their relationship.
\end{document}
